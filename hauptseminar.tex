\documentclass[12pt,ngerman]{scrartcl}
\usepackage{babel}
\usepackage[utf8]{inputenc}
\usepackage[T1]{fontenc}
\usepackage{lmodern}
\usepackage{amsmath,amssymb,amsfonts,amsthm}
\usepackage{bbm}
\usepackage{wasysym}
\pagestyle{empty}
\DeclareMathOperator{\ord}{ord}
\begin{document}
Zunächst wollen wir uns noch etwas mit der Norm $|\cdot|_p$ beschäftigen. Wir sagen $a \equiv b \pmod {p^i}$, wenn $b-a=qp^i$ mit $q=\frac{n}{m}$ (teilerfremde Darstellung) und $p \nmid m$. Dann gilt  $|a-b|_p=|q|_p|p^i|_p = \frac{|n|_p}{|m|_p}p^{-i} \overset{p \nmid m}{=} |n|_p p^{-i} \le p^{-i}$. Sei andererseits $|a-b|_p \le p^{-i}$, mit $a-b=x/y$ (teilerfremde Darstellung). $\Rightarrow 1/p^i \ge |a-b|_p = 1/{p^{\ord_p(x)-\ord_p(y)}}$ $\Rightarrow i \le \ord_p(x)-\ord_p(y)$ $\Rightarrow \ord_p(x) \ge \ord_p(y)+i$ $\Rightarrow x=p^{i}p^{\ord_p(y)}k$mit $k \in \mathbb{Q}$ ohne p im Nenner. $\Rightarrow a-b=x/y=p^i(\frac{\ord_p(y)}{y}k)$. $\frac{\ord_p(y)}{y}$ ist eine rationale Zahl, die kein p im Nenner hat, dasselbe gilt für k, d.h. wir haben insgesamt wieder dieselbe Darstellung wie oben. (6)\\

Wir wollen nun von dem abstrakten Begriff von ''Äquivalenzklassen von Cauchy-Folgen rationaler Zahlen'' wegkommen und eine einfachere Darstellung der p-adischen Zahlen finden. Dazu beweisen wir folgenden Satz:\\[3 mm]
\textbf{Satz 2} \\
\textit{Jedes $a \in \mathbb{Q}_p$ mit $|a|_p \le 1$ hat genau einen Repräsentanten $(a_i)$, für den für alle $i \ge 1$ gilt: $0 \le a_i < p^i$ und $a_i \equiv a_{i+1} \pmod {p^i}$}. \\[1 mm]
\textit{Beweis.} Eindeutigkeit: \textit{Annahme: Es existieren Repräsentanten $(a_i)\neq (a_i')$ von a, die die Bedingungen erfüllen.}\\ Für $a_{i_0} \neq a_{i_0}'$ gilt dann $a_{i_0} \not\equiv a_{i_0}' \pmod{p^{i_0}}$. \\ \textbf{Diese Stelle habe ich leider nicht verstanden. Ist meine Definition oben, was modulo in den rationalen Zahlen bedeutet richtig? Man könnte ja z.B. $p=3,i_0=1,a_0=0,a_0'=1.5$ haben, dann wäre mit der oberen Definition $a_0 \equiv a_0'$.}\\ Wir haben $a_{i_0} \equiv a_{i_0+1} \pmod {p^{i_0}}, a_{i_0+1} \equiv a_{i_0+2} \pmod {p^{i_0+1}}$, also auch $a_{i_0} \equiv a_{i_0+2} \pmod {p^{i_0}}$, induktiv folgt damit für alle $i \ge i_0$: $a_i\equiv a_{i_0} \not\equiv a_{i_0}'\equiv a_i' \pmod{p^{i_0}}$. $\Rightarrow (6): |a_{i}-a_{i}'|_p > p^{-i_0}$ $\Rightarrow (a_i) \nsim (a_i')$ \lightning \\
Für die Existenz brauchen wir zunächst noch ein Lemma: \\[1 mm]
\textbf{Lemma 3} \\
\textit{Für $x \in \mathbb{Q}$ mit $|x|_p \le 1$ gilt: $\forall i \in \mathbb{N} \exists \alpha \in \mathbb{Z}$ sodass $|\alpha - x|_p \le p^{-i}$. Wir können $\alpha$ sogar aus der Menge $\{0,1, \ldots, p^i-1\}$ wählen.} \\
\textit{Beweis.} Sei $i \in \mathbb{N}$, $\frac{a}{b}$ (teilerfr.), $|x|_p \le 1$ \\
$\Rightarrow (6): p \nmid b$ $\overset{\text{p prim}}{\Rightarrow} p^i,b$ sind teilerfremd $\Rightarrow \exists n,m \in \mathbb{Z}: bm+p^in=1$. Wir setzen $\alpha':=am$, dann haben wir \[ |\alpha'-x|_p = |am-a/b|_p = |x|_p|bm-1|_p=|bm-1|_p\le|p^in|_p=p^{-i}|n|_p\le p^{-i}.\]
Wir erhalten $\alpha \in \{0,1, \ldots, p^i-1\}$ durch $\alpha:=\alpha'+zp^i$ mit geeignetem $z\in \mathbb{Z}$, dann haben wir: $|\alpha-x|_p=|\alpha'-x+zp^i|_p \le \max\{|\alpha'-x|_p,|zp^i|_p\}$. Der erste Term ist wie gesehen $\le p^{-i}$ und $|zp^i|_p=|z|_pp^{-i}\le p^{-i}$. \hfill $\square$\\
Damit können wir nun die Existenz beweisen:\\
\textit{Fortsetzung Beweis Satz 2.} Sei $(b_i) \in S$ mit $|\overline{(b_i)}|_p\le 1$. Wir wählen für $j \in \mathbb{N}$ $N(j) \in \mathbb{N}$, sodass $|b_i-b_{i'}|_p\le p^{-j} \forall i,i'\ge N(j)$(7). Wir können $N(j)$ oBdA streng monoton steigend wählen(insb. $N(j)\ge j \forall j$). Für $i,i' \ge N(1): |b_i|_p \le \max\{|b_{i'}|_p,|b_i-b_{i'}|_p\} \le \max\{|b_{i'}|_p,1/p\}$ Wenn wir $i'$ gegen unendlich laufen lassen geht $|b_{i'}|_p$ gegen $|a|_p\le 1$, d.h. wir haben $|b_i|\le 1$. Nun können wir Lemma 3 anwenden: $\forall j \in \mathbb{N} \exists 0 \le a_j < p^{j}: |a_j-b_{N(j)}|_p\le p^{-j}$ (8).\\
$(a_i)$ soll natürlich unser Repräsentant von $\overline{(b_i)}$ sein. Zunächst zeigen wir die zweite gewünschte Eigenschaft: \[|a_{i+1}-a_i|_p\le \max\{\underbrace{|a_{i+1}-b_{N(i+1)}|_p}_{(8): \le p^{-(i+1)}},\underbrace{|b_{N(i+1)}-b_{N(i)}|_p}_{(7): \le p^{-i}},\underbrace{|a_i-b_{N(i)}|_p}_{(8): \le p^{-i}}\}\le p^{-i}\] $\Rightarrow (6): a_{i+1} \equiv a_i \pmod {p^i}$.\\
Es bleibt noch zu zeigen, dass $(a_i)$ überhaupt ein Repräsentant dieser Klasse ist: Wähle dazu $i\ge N(j)\ge j$. Dann ist - wie eben gezeigt - $a_i \equiv a_j \pmod p^j$ $\Rightarrow |a_i-a_j|_p \le p^{-j}$(9). Damit gilt: \[|a_i-b_i|_p\le\max\{\underbrace{|a_i-a_j|_p}_{(9): \le p^{-j}},\underbrace{|a_j-b_{N(j)}|_p}_{(8): \le p^{-j}},\underbrace{|b_{N(j)}-b_i|_p}_{(7): \le p-j}\} \le p^{-j}\]
Wenn i gegen $\infty$ geht, geht auch j gegen $\infty$ und wir haben $|a_i-b_i|_p \rightarrow 0$, also $(a_i) \sim (b_i)$. \hfill $\square$


\end{document}